\documentclass{beamer}
\usetheme{CambridgeUS}
\usepackage{luatexja}
\usepackage{listings}
\lstset{
  numbers=left,
  basicstyle=\ttfamily,
}
\usepackage{color}
\usepackage{amsmath,amssymb}
\usepackage{bm}
\usepackage{physics}
\usepackage{comment}
\usepackage{autobreak}
\usepackage{mathtools}
\usepackage{mathcommand}
\mathtoolsset{showonlyrefs=true}
\usepackage{graphicx}
\usepackage{here}
\usepackage{tikz}
\usepackage{hyperref}

\title{非平衡統計力学}
\author{Toroid153846}
\institute{TUS 物理学科 B4}
\date{\today}

\begin{document}

\begin{frame}
  \titlepage
\end{frame}
\begin{frame}
    {\Large 目次}
     \tableofcontents
\end{frame}
\section{マルコフジャンプ過程}
\subsection{セットアップ}
\begin{frame}{系の設定}
\begin{itemize}
  \item 着目系と熱浴$\nu=1,2,\dots$
  \item 時間$0$から$\tau$までの時間発展
  \item 操作プロトコル
  \begin{align}
    \bm{\lambda}_\tau=\{\lambda(t)\}_{t=0}^\tau
  \end{align}
  \item 状態の経路
  \begin{align}
    \bm{x}_\tau=\{x(t)\}_{t=0}^\tau
  \end{align}
\end{itemize}
\end{frame}

\begin{frame}[fragile]{経路の定義}
    時間$t_k$の状態の変化は熱浴$\nu_k$により起こる
\begin{center}
\begin{tikzpicture}
  \draw[->,thick] (0,0) -- (0,5) node[above] {$x$};
  \draw[->,thick] (0,0) -- (10.5,0) node[right] {$t$};
  \draw[thick] (3,1) -- (0,1) node[left] {$x_0$};
  \draw[dashed] (3,4) -- (0,4) node[left] {$x_1$};
  \draw[thick] (3,4) -- (5,4);
  \draw[dashed] (5,2) -- (0,2) node[left] {$x_2$};
  \draw[thick] (5,2) -- (6,2);
  \draw[dashed] (9,3) -- (0,3) node[left] {$x_K$};
  \draw[thick] (10,3) -- (9,3);
  \draw[dashed] (10,3) -- (10,0) node[below] {$\tau$};
  \draw[dashed] (0,1) -- (0,0) node[below] {$0$};
  \draw[dashed] (3,4) -- (3,0) node[below] {$t_1$};
  \draw[dashed] (5,4) -- (5,0) node[below] {$t_2$};
  \draw[dashed] (6,0) -- (7,0) node[below] {$\cdots$};
  \draw[dashed] (9,5) -- (9,0) node[below] {$t_K$};
\end{tikzpicture}
\end{center}
\end{frame}

\begin{frame}{物理量の定義}
    \begin{itemize}
        \item 確率分布
        \begin{align}
            P(x,t)
        \end{align}
        \item 遷移確率密度
        \begin{align}
            R_\nu(x'|x;t)
        \end{align}
        \item 確率流
        \begin{align}
            J_\nu(x'\mid x;t):=R_\nu(x'\mid x;t)P(x,t)-R_\nu(x\mid x';t)P(x',t)
        \end{align}
        \item エスケープレート
        \begin{align}
            \gamma(x,t):=\sum_{\nu,x'(\neq x)}R_\nu(x'\mid x;t)
        \end{align}
    \end{itemize}
\end{frame}

\begin{frame}[fragile]{経路積分と期待値と規格化}
経路に対応する物理量$A$の期待値
\begin{align*}
    \ev{A}=&\int\mathcal{D}\bm{x}_\tau A[\bm{x}_\tau,\bm{\lambda}_\tau]P[\bm{x}_\tau\mid\bm{\lambda}_\tau]\\
    =&\sum_{K=0}^{\infty}\sum_{\lbrace x_k\rbrace_{k=0}^K}\sum_{\lbrace\nu_k\rbrace_{k=1}^K}\int_{0<t_1<\cdots<t_K<\tau}A[\bm{x}_\tau,\bm{\lambda}_\tau]\\
    &\times P(x_0,0)\exp\left(-\int_0^\tau \dd{t'}\gamma\left(x(t'),t'\right)\right)\prod_{k=1}^K \left(R_{\nu_k}\left(x_k\mid x_{k-1};t_k\right)\dd{t_k}\right)
\end{align*}
規格化条件
\begin{align}
    \ev{1}=&1
\end{align}
\end{frame}
\subsection{「流れ」を表す物理量}
\begin{frame}{定義と経路の関数}
    \begin{block}{「流れ」を表す物理量$D$の定義}
      熱浴 $\nu$ と2つの状態 $x,x'$ に対応する物理量 $D^\nu_{x',x}$ であり,以下の反対称性を満たす。
      \[
        D^\nu_{x',x} = -\,D^\nu_{x,x'}.
      \]
    \end{block}
    物理量$D$による経路の関数
    \begin{align}
      \hat{\mathcal{J}}_D[\bm{x}_\tau,\bm{\lambda}_\tau]:=\sum^K_{k=1}D^{\nu_k}_{x_k,x_{k-1}}\\
    \end{align}
\end{frame}
\begin{frame}[fragile]{関連する経路の関数}
    時間$t$に依存した経路の関数
    \begin{align}
        \hat{\mathcal{J}}_D[\bm{x}_\tau,\bm{\lambda}_\tau](t)=\sum_{\lbrace k|t_k \le t\rbrace}D^{\nu_k}_{x_k,x_{k-1}}\\
    \end{align}
    経路の関数の時間微分
    \begin{align}
        \hat{\dot{\mathcal{J}}}_D[\bm{x}_\tau,\bm{\lambda}_\tau](t)=\sum_{k=1}^K D^{\nu_k}_{x_{k},x_{k-1}}\delta(t-t_k)\label{eq:dotJ}
    \end{align}
\end{frame}
\begin{frame}[fragile]{期待値}
    経路の関数の時間微分の期待値$J_D(t)$
    \begin{align}
        J_D(t)=&\ev*{\hat{\dot{\mathcal{J}}}_D(t)}\\
        =&\sum_{K=0}^{\infty}\sum_{\lbrace x_k\rbrace_{k=0}^K}\sum_{\lbrace\nu_k\rbrace_{k=1}^K}\int_{0<t_1<\cdots<t_K<\tau}\hat{\dot{\mathcal{J}}}_D[\bm{x}_\tau,\bm{\lambda}_\tau](t)\\
        &\times P(x_0,0)\exp\left(-\int_0^\tau \dd{t'}\gamma\left(x(t'),t'\right)\right)\prod_{k=1}^K \left(R_{\nu_k}\left(x_k\mid x_{k-1};t_k\right)\dd{t_k}\right)\\
        =&\sum_{\nu,x'\neq x}R_{\nu}\left(x'\mid x;t\right)P(x,t)D^{\nu}_{x',x}\\
    \end{align}
\end{frame}
\begin{frame}[fragile]{「流れ」を表す物理量の具体例}
    ある2つの状態$x_0,x'_0$
    \begin{align}
        D^{\nu}_{x,x'}=&
        \begin{cases}
            1 & ((x,x')=(x_0,x'_0))\\
            -1 & ((x,x')=(x'_0,x_0))\\
            0 & (\text{else})
        \end{cases}\\
    \end{align}
    として
    \begin{align}
      J_D(t)=&\sum_{\nu,x'\neq x}R_{\nu}\left(x'\mid x;t\right)P(x,t)D^{\nu}_{x',x}\\
      =&\sum_\nu\left( R_{\nu}\left(x'_0\mid x_0;t\right)P(x_0,t)-R_{\nu}\left(x_0\mid x'_0;t\right)P(x'_0,t) \right)\\
      =&\sum_\nu J_\nu(x'_0\mid x_0;t)\\
    \end{align}
\end{frame}
\begin{frame}[fragile]{「流れ」を表す物理量の具体例}
    状態$x$に対応する物理量$A_x$
    \begin{align}
        D^\nu_{x',x}=A_{x'}-A_{x}
    \end{align}
    として
    \begin{align}
      J_D(t)=&\sum_{\nu,x'\neq x}R_{\nu}\left(x'\mid x;t\right)P(x,t)\left( A_{x'}-A_x \right) \\
      =&\sum_{\nu,x\neq x'}J_\nu(x'_0\mid x_0;t)A_{x'}\\
      =&\dv{A(t)}{t}
    \end{align}
\end{frame}
\subsection{熱力学不確定性関係}
\begin{frame}[fragile]{熱力学不確定性関係の主張}
    確率分布$P(x,t)=P(x)$,遷移確率密度$R(x',x;t)=R(x',x)$の場合
    \begin{align}
        \sigma\ge 2\frac{\ev*{\hat{\mathcal{J}}_D}^2}{\ev*{\Delta\hat{\mathcal{J}}_D^2}}
    \end{align}
    一般の場合($\lambda=\lambda(vt)$)
    \begin{align}
        \sigma\ge 2\frac{(\tau J_D(\tau)-v\pdv{v}\ev*{\hat{\mathcal{J}}_D})^2}{\ev*{\Delta\hat{\mathcal{J}}_D^2}}
        \label{eq:thermal_uncertain_general}
      \end{align}
\end{frame}

\end{document}
