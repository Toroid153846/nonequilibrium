\documentclass{ltjsarticle}


%ソースコード
\usepackage{listings}
\lstset{
  numbers=left,
  basicstyle=\ttfamily,
}
%装飾
\usepackage{color}
% 数式
\usepackage{amsmath,amssymb}
\usepackage{bm}
\usepackage{physics}
\usepackage{comment}
\usepackage{autobreak}
\usepackage{mathtools}
\usepackage{mathcommand}
\mathtoolsset{showonlyrefs=true}
% 数式処理
\usepackage{luacas}
% 画像
\usepackage{graphicx}
\usepackage{here}
\usepackage{tikz}
% 引用
\usepackage{hyperref}

\title{非平衡統計力学}
\author{Toroid153846}
\date{\today}

\begin{document}
\maketitle


  \begin{align}
    \dot{\sigma}_{ex}=&\sum_{x\neq x'}\sum_\nu R_\nu (x'|x;t)P(x,t)\log\frac{R_\nu (x'|x;t)P(x,t)}{\tilde{R}_\nu (x|x';t)P(x',t)}\\
    \dot{\sigma}_{ex}\left[ \bm{x}_\tau,\bm{\lambda}_\tau \right] =&\sum^K_{k=1}\log\frac{R_{\nu_k} (x_k|x_{k-1};t)P(x_\tau,0)}{\tilde{R}_{\nu_k} (x_{k-1}|x_k;t)P(x_0,\tau)}\\
  \end{align}
  と定義して
  \begin{align}
    \int dt \dot{\sigma}_{ex}=\int D\bm{x}_\tau \dot{\sigma}_{ex}\left[ \bm{x}_\tau,\bm{\lambda}_\tau \right] P\left[ \bm{x}_\tau|\bm{\lambda}_\tau \right]\\
  \end{align}
が成り立つことを示してください。

\end{document}