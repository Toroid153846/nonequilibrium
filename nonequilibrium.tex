\documentclass{ltjsarticle}


%ソースコード
\usepackage{listings}
\lstset{
  numbers=left,
  basicstyle=\ttfamily,
}
%装飾
\usepackage{color}
% 数式
\usepackage{amsmath,amssymb}
\usepackage{bm}
\usepackage{physics}
\usepackage{comment}
\usepackage{autobreak}
\usepackage{mathtools}
\usepackage{mathcommand}
\mathtoolsset{showonlyrefs=true}
% 数式処理
\usepackage{luacas}
% 画像
\usepackage{graphicx}
\usepackage{here}
\usepackage{tikz}
% 引用
\usepackage{hyperref}

\title{非平衡統計力学}
\author{Toroid153846}
\date{\today}

\begin{document}
\maketitle


%   \begin{align}
%     \dot{\sigma}_{ex}=&\sum_{x\neq x'}\sum_\nu R_\nu (x'|x;t)P(x,t)\log\frac{R_\nu (x'|x;t)P(x,t)}{\tilde{R}_\nu (x|x';t)P(x',t)}\\
%     \dot{\sigma}_{ex}\left[ \bm{x}_\tau,\bm{\lambda}_\tau \right] =&\sum^K_{k=1}\log\frac{R_{\nu_k} (x_k|x_{k-1};t)P(x_\tau,0)}{\tilde{R}_{\nu_k} (x_{k-1}|x_k;t)P(x_0,\tau)}\\
%   \end{align}
%   と定義して
%   \begin{align}
%     \int dt \dot{\sigma}_{ex}=\int D\bm{x}_\tau \dot{\sigma}_{ex}\left[ \bm{x}_\tau,\bm{\lambda}_\tau \right] P\left[ \bm{x}_\tau|\bm{\lambda}_\tau \right]\\
%   \end{align}
% が成り立つことを示してください。\\
\section{マルコフジャンプ過程}
\subsection{基本的なセットアップ}
\subsubsection{系の設定}
着目系とその系と相互作用する熱浴$\nu=1,2,\dots$がある。\\
マルコフジャンプ過程での着目系の時間$0$から$\tau$までの時間発展を考える。\\
操作プロトコル$\bm{\lambda}_\tau=\{\lambda(t)\}_{t=0}^\tau$は操作パラメーター(磁場などエネルギー準位を変化させる)の時間変化を示す。\\
下のように時間$t_k$($k=1,\dots,K$であり$0<t_1<t_2<\cdots<t_K<\tau$を満たす)において熱浴$\nu_{k}$による着目系の状態の遷移($x_{k-1}\to x_k$)が$K$回起こった経路を$\bm{x}_\tau=\lbrace x(t)\rbrace_{t=0}^\tau$とする。\\
\begin{tikzpicture}
  %\draw (0,0) grid (10,5);
  \draw[->,thick] (0,0) -- (0,5) node[above] {$x$};
  \draw[->,thick] (0,0) -- (10.5,0) node[right] {$t$};
  \draw[thick] (3,1) -- (0,1) node[left] {$x_0$};
  \draw[dashed] (3,4) -- (0,4) node[left] {$x_1$};
  \draw[thick] (3,4) -- (5,4);
  \draw[dashed] (5,2) -- (0,2) node[left] {$x_2$};
  \draw[thick] (5,2) -- (6,2);
  \draw[dashed] (9,3) -- (0,3) node[left] {$x_K$};
  \draw[thick] (10,3) -- (9,3);
  \draw[dashed] (10,3) -- (10,0) node[below] {$\tau$};
  \draw[dashed] (0,1) -- (0,0) node[below] {$0$};
  \draw[dashed] (3,4) -- (3,0) node[below] {$t_1$};
  \draw[dashed] (5,4) -- (5,0) node[below] {$t_2$};
  \draw[dashed] (6,0) -- (7,0) node[below] {$\cdots$};
  \draw[dashed] (9,5) -- (9,0) node[below] {$t_K$};
\end{tikzpicture}
\subsubsection{いくつかの物理量の定義}
時刻$t$において状態$x$にある確率を確率分布$P(x,t)$とする。\\
最初に状態$x$にあったときに$x'$に時間$t$から$t+\dd{t}$の間に熱浴$\nu$により遷移する単位時間当たりの確率を遷移確率$R_\nu(x'|x;t)$とする。\\
熱浴$\nu$によって$x$から$x'$へ流れる単位時間当たりの確率を確率流といい、
\begin{align}
  J_\nu(x'\mid x;t):=R_\nu(x'\mid x;t)P(x,t)-R_\nu(x\mid x';t)P(x',t)
\end{align}
と定義する。\\
状態$x$にあるときに単位時間あたりに$x$以外の状態に遷移する確率をエスケープレート$\gamma(x,t)$といい、
\begin{align}
  \gamma(x,t)=\sum_{\nu,x'(\neq x)}R_\nu(x'\mid x;t)
\end{align}\\
\subsubsection{経路確率と経路積分の導出}
まず経路積分の定義から(経路積分の定義はゆらぐ系の熱力学 齊藤 圭司から)\\
\begin{align}
  \sum_{K=0}^{\infty}\sum_{\lbrace x_k\rbrace_{k=0}^K}\sum_{\lbrace\nu_k\rbrace_{k=1}^K}\int_{0<t_1<\cdots<t_K<\tau}\\
\end{align}
なお、条件$x_{k-1}\neq x_k$は満たす。この経路積分は系の設定において考えられる全ての経路を考慮していることは確かめられる。\\
経路の確率を導出する。最初の状態において$x_0$にある確率は$P(x_0,0)$である。また、時間$t_{k}$から$t_{k+1}$において一定値$x_k$を取る確率は、エスケープレート$\gamma(x,t)$を用いれば時間$t$から$t+\Delta{t}$において状態$x$のままである確率は$1-\gamma(x,t)\Delta{t}$
\begin{align}
  \prod_{t=t_k}^{t_{k+1}}(1-\gamma(x_k,t)\Delta{t})=&\prod_{t=t_k}^{t_{k+1}}\exp\left(-\gamma(x_k,t)\Delta{t}\right)\\
  =&\exp\left(-\int_{t_k}^{t_{k+1}}\gamma(x_k,t)\dd{t}\right)\\
\end{align}
であり、時間$t_k$において状態$x_{k-1}$から$x_k$に遷移する確率は$R_{\nu_k}(x_k|x_{k-1};t_k)\dd{t}$である。(ただ本当は規格化係数が必要なのかもしれない)\\
これらをすべて掛けて計算すると
\begin{align}
  P[\bm{x}_\tau\mid\bm{\lambda}_\tau]
  =&P(x_0,0)\prod_{k=1}^K \left( \exp\left(-\int_{t_k}^{t_{k+1}}\gamma(x_k,t')\dd{t'}\right)R_{\nu_k}(x_k|x_{k-1};t_k) \dd{t_k}\right)\exp\left(-\int_{t_K}^{\tau}\gamma(x_k,t')\dd{t'}\right) \\
  =&P(x_0,0)\exp\left(-\int_0^\tau \dd{t'}\gamma\left(x(t'),t'\right)\right)\prod_{k=1}^K \left(R_{\nu_k}\left(x_k\mid x_{k-1};t_k\right)\dd{t_k}\right)
\end{align}
このように経路の確率が求まる。\\
そのため、経路で定義される物理量$A[\bm{x}_\tau,\bm{\lambda}_\tau]$の期待値$\ev{A}$は次のように表される。
\begin{align}
  \ev{A}=&\int \mathcal{D}\bm{x}_\tau A[\bm{x}_\tau,\bm{\lambda}_\tau]P[\bm{x}_\tau\mid\bm{\lambda}_\tau]\\
  =&\sum_{K=0}^{\infty}\sum_{\lbrace x_k\rbrace_{k=0}^K}\sum_{\lbrace\nu_k\rbrace_{k=1}^K}\int_{0<t_1<\cdots<t_K<\tau}\dd{t_1}\cdots \dd{t_K}A[\bm{x}_\tau,\bm{\lambda}_\tau]\\
  &\times P(x_0,0)\exp\left(-\int_0^\tau \dd{t'}\gamma\left(x(t'),t'\right)\right)\prod_{k=1}^K R_{\nu_k}\left(x_k\mid x_{k-1};t_k\right)\\
\end{align}
また、初期分布に関係なく
\begin{align}
  \ev{1}=&\sum_{K=0}^{\infty}\sum_{\lbrace x_k\rbrace_{k=0}^K}\sum_{\lbrace\nu_k\rbrace_{k=1}^K}\int_{0<t_1<\cdots<t_K<\tau}\dd{t_1}\cdots \dd{t_K}\\
  &\times P(x_0,0)\exp\left(-\int_0^\tau \dd{t'}\gamma\left(x(t'),t'\right)\right)\prod_{k=1}^K R_{\nu_k}\left(x_k\mid x_{k-1};t_k\right)\\
  =&1
\end{align}
が成り立つのは確率の規格化条件からわかる。
\subsection{「流れ」を表す物理量}
\subsubsection{定義と関連した物理量}
2つの状態$x$と$x'$と熱浴$\nu$において定義される「流れ」を表す物理量$D^\nu_{x',x}$について\\
\begin{align}
  D^\nu_{x',x}=-D^\nu_{x,x'}
\end{align}
のような反対称性を満たす。\\
ここで、経路$\bm{x}_\tau$においてその物理量$D$による経路の関数を次のように定義する。
\begin{align}
  \hat{\mathcal{J}}_D[\bm{x}_\tau,\bm{\lambda}_\tau]=\sum^K_{k=1}D^{\nu_k}_{x_k,x_{k-1}}\\
\end{align}
定義からわかる通り実際には経路$\bm{x}_\tau$には状態の遷移($x_{k-1}\to x_k$)の原因である熱浴$\nu_k$の情報が含まれている。\\
ここで、この経路の関数を時間$t$の関数として表すと
\begin{align}
  \hat{\mathcal{J}}_D[\bm{x}_\tau,\bm{\lambda}_\tau](t)=\sum_{\lbrace k|t_k \le t\rbrace}D^{\nu_k}_{x_k,x_{k-1}}\\
\end{align}
したがって、
\begin{align}
  \hat{\dot{\mathcal{J}}}_D[\bm{x}_\tau,\bm{\lambda}_\tau](t)=\sum_{k=1}^K D^{\nu_k}_{x_{k},x_{k-1}}\delta(t-t_k)\label{eq:dotJ}
\end{align}
として時間微分を定義できる。\\
\eqref{eq:dotJ}式を用いて、流れの積分$\hat{\mathcal{J}}_D$の時間微分の経路積分した値$J_D(t)$を求めてみる。\\
経路積分と確率の定義を用いて、
\begin{align}
  J_D(t)=&\ev*{\hat{\dot{\mathcal{J}}}_D(t)}\\
  =&\int \mathcal{D}\bm{x}_\tau \hat{\dot{\mathcal{J}}}_D[\bm{x}_\tau,\bm{\lambda}_\tau](t)P[\bm{x}_\tau|\bm{\lambda}_\tau]\\
  =&\sum_{K=0}^{\infty}\sum_{\lbrace x_k\rbrace_{k=0}^K}\sum_{\lbrace\nu_k\rbrace_{k=1}^K}\int_{0<t_1<\cdots<t_K<\tau}\dd{t_1}\cdots \dd{t_K}\sum_{k'=1}^K D^{\nu_{k'}}_{x_{k'},x_{k'-1}}\delta(t-t_{k'})\\
  \times&P(x_0,0)\exp\left(-\int_0^\tau \dd{t'}\gamma\left(x(t'),t'\right)\right)\prod_{k=1}^K R_{\nu_k}\left(x_k\mid x_{k-1};t_k\right)\\
  =&\sum_{K=0}^{\infty}\sum_{k'=1}^K\sum_{\lbrace x_k\rbrace_{k=0}^K}\sum_{\lbrace\nu_k\rbrace_{k=1}^K}\int_{0<t_1<\cdots<t_{k'-1}<t<t_{k'+1}<\cdots<t_K<\tau}\dd{t_1}\cdots\dd{t_{k'-1}}\dd{t_{k'+1}}\cdots \dd{t_K} D^{\nu_{k'}}_{x_{k'},x_{k'-1}}\\
  \times&P(x_0,0)\exp\left(-\int_0^\tau \dd{t'}\gamma\left(x(t'),t'\right)\right)\prod_{1\le k\le K,k(\neq k')} R_{\nu_{k}}\left(x_{k}\mid x_{k-1};t_k\right)R_{\nu_{k'}}\left(x_{k'}\mid x_{k'-1};t\right)\\
  =&\sum_{K=0}^{\infty}\sum_{k'=1}^K\sum_{\lbrace x_k\rbrace_{k=0}^{k'}}\sum_{\lbrace\nu_k\rbrace_{k=1}^{k'}}\int_{0<t_1<\cdots<t_{k'-1}<t}\dd{t_1}\cdots\dd{t_{k'-1}}D^{\nu_{k'}}_{x_{k'},x_{k'-1}} \\
  \times&P(x_0,0)\exp\left(-\int_0^t \dd{t'}\gamma\left(x(t'),t'\right)\right)\prod_{k=1}^{k'-1} R_{\nu_k}\left(x_k\mid x_{k-1};t_k\right)R_{\nu_{k'}}\left(x_{k'}\mid x_{k'-1};t\right)\\
  \times&\sum_{\lbrace x_k\rbrace_{k=k'+1}^{K}}\sum_{\lbrace\nu_k\rbrace_{k=k'+1}^{K}}\int_{t<t_{k'+1}<\cdots<t_K<\tau}\dd{t_{k'+1}}\cdots \dd{t_K}\\
  \times&\exp\left(-\int_t^\tau \dd{t'}\gamma\left(x(t'),t'\right)\right)\prod_{k=k'+1}^K R_{\nu_k}\left(x_k\mid x_{k-1};t_k\right)\\
  =&\sum_{K=0}^{\infty}\sum_{k'=1}^K\sum_{\lbrace x_k\rbrace_{k=0}^{k'}}\sum_{\lbrace\nu_k\rbrace_{k=1}^{k'}}\int_{0<t_1<\cdots<t_{k'-1}<t}\dd{t_1}\cdots\dd{t_{k'-1}}D^{\nu_{k'}}_{x_{k'},x_{k'-1}} \\
  \times&P(x_0,0)\exp\left(-\int_0^t \dd{t'}\gamma\left(x(t'),t'\right)\right)\prod_{k=1}^{k'-1} R_{\nu_k}\left(x_k\mid x_{k-1};t_k\right)R_{\nu_{k'}}\left(x_{k'}\mid x_{k'-1};t\right)\\
  =&\sum_{K=0}^{\infty}\sum_{k'=1}^K\sum_{\lbrace x_k\rbrace_{k=0}^{k'}}\sum_{\lbrace\nu_k\rbrace_{k=1}^{k'}}\int_{0<t_1<\cdots<t_{k'-1}<t}\dd{t_1}\cdots\dd{t_{k'-1}} \\
  \times&P(x_0,0)\exp\left(-\int_0^t \dd{t'}\gamma\left(x(t'),t'\right)\right)\prod_{k=1}^{k'-1} R_{\nu_k}\left(x_k\mid x_{k-1};t_k\right)\\
  \times&D^{\nu_{k'}}_{x_{k'},x_{k'-1}}R_{\nu_{k'}}\left(x_{k'}\mid x_{k'-1};t\right)\\
  =&\sum_{x}P(x,t)\sum_{\nu,x'(\neq x)}D^{\nu}_{x',x}R_{\nu}\left(x'\mid x;t\right)\\
  =&\sum_{\nu,x'\neq x}R_{\nu}\left(x'\mid x;t\right)P(x,t)D^{\nu}_{x',x}
  \label{eq:path_integral_JD}
\end{align}
この式の計算に関して、上から2行目から3,4行目は定義を代入し、3,4行目から5,6行目は$t_{k'}$によるデルタ関数の積分を行って$t_k$と$t$が入れ替え、5,6行目から7,8,9,10行目は定数である$t$を用いて積分区間を分け、7,8,9,10行目から11,12行目では9,10行目が$0$から$\tau$の積分経路を$t$から$\tau$までで切り取った経路について$1$を積分しているので初期状態である$x_k$がどんな値を取っても$1$となるため9,10行目に相当する部分が消えている。13,14,15行目から16行目は13,14行目が時刻$t$において状態$x_{k'-1}$を取るような確率であるためこのような結果となる。\\
これは確かに文献に書いてあった通りの表式である。この$J_D(t)$に関して「流れ」を表す物理量の具体例を見てみる。
\subsubsection{具体例}
1つ目\\
特定の2つの状態$x_0,x'_0(x_0\neq x'_0)$として、$D^{\nu}_{x'_0,x_0}=-D^{\nu}_{x_0,x'_0}=1$であり、他は$D^{\nu}_{x,x'}=0$であるとする。\\
このとき、$J_D(t)$は次のように表される。
\begin{align}
  J_D(t)=&\sum_{\nu,x'\neq x}R_{\nu}\left(x'\mid x;t\right)P(x,t)D^{\nu}_{x',x}\\
  =&\sum_\nu\left( R_{\nu}\left(x'_0\mid x_0;t\right)P(x_0,t)-R_{\nu}\left(x_0\mid x'_0;t\right)P(x'_0,t) \right)\\
  =&J(x'_0\mid x_0;t)\\
\end{align}
よって、$x_0$から$x'_0$への確率流を表している。\\
2つ目\\
状態$x$に対して定義されるある物理量$A_x$に対して,$D^\nu_{x',x}=A_{x'}-A_x$と定義すると反対称性を満たす。
このとき、$J_D(t)$は次のように表される。
\begin{align}
  J_D(t)=&\sum_{\nu,x'\neq x}R_{\nu}\left(x'\mid x;t\right)P(x,t)D^{\nu}_{x',x}\\
  =&\sum_{\nu,x'\neq x}R_{\nu}\left(x'\mid x;t\right)P(x,t)\left( A_{x'}-A_x \right) \\
  =&\sum_{\nu,x'\neq x}R_{\nu}\left(x'\mid x;t\right)P(x,t)A_{x'}-\sum_{\nu,x'\neq x}R_{\nu}\left(x'\mid x;t\right)P(x,t)A_x \\
  =&\sum_{\nu,x'\neq x}R_{\nu}\left(x'\mid x;t\right)P(x,t)A_{x'}-\sum_{\nu,x\neq x'}R_{\nu}\left(x\mid x';t\right)P(x',t)A_{x'} \\
  =&\sum_{\nu,x'\neq x}\left( R_{\nu}\left(x'\mid x;t\right)P(x,t)- R_{\nu}\left(x\mid x';t\right)P(x',t)\right) A_{x'}\\
  =&\sum_{x'}\sum_{\nu,x(\neq x')}\left( R_{\nu}\left(x'\mid x;t\right)P(x,t)- R_{\nu}\left(x\mid x';t\right)P(x',t)\right) A_{x'}\\
  =&\sum_{x'}\pdv{P(x',t)}{t}A_{x'}\\
  =&\dv{t}\left( \sum_{x'}P(x',t)A_{x'} \right) \\
  =&\dv{A(t)}{t}
\end{align}
この式の上から6行目から7行目はマスター方程式を用いている。また、$A(t)$は物理量$A_x$の期待値とした。\\
\subsection{熱力学不定性関係}
\subsubsection{主張}
熱力学不定性関係の主張は次のように表される。\\
確率分布$P(x,t)$と遷移確率密度$R(x',x;t)$は時間に依存しないとする。(熱平衡状態とは限らない、定常状態である)
\begin{align}
  \sigma\ge 2\frac{\ev*{\hat{\mathcal{J}}_D}^2}{\ev*{\Delta\hat{\mathcal{J}}_D^2}}
  \label{eq:thermal_uncertain_steady}
\end{align}
時間に依存する一般の場合については、操作プロトコル$\bm{\lambda}=\bm{\lambda}(vt)$として
\begin{align}
  \sigma\ge 2\frac{(\tau J_D(\tau)-v\pdv{v}\ev*{\hat{\mathcal{J}}_D})^2}{\ev*{\Delta\hat{\mathcal{J}}_D^2}}
  \label{eq:thermal_uncertain_general}
\end{align}
\subsubsection{短時間極限}
短時間極限$\tau\to 0$の場合について考える
\begin{align}
  \ev*{\Delta\hat{\mathcal{J}}_D^2}=&\ev*{\hat{\mathcal{J}}_D^2}-\ev*{\hat{\mathcal{J}}_D}^2\\
  =&\ev*{\hat{\mathcal{J}}_D^2}-\tau^2\ev*{\hat{\dot{\mathcal{J}}}_D}^2+\order{\tau^3}\\
  =&\tau\ev{\dv{(\hat{\mathcal{J}}_D^2)}{t}}-\tau^2\ev*{\hat{\dot{\mathcal{J}}}_D}^2+\order{\tau^3}
  \label{eq:fluctuation_relation_pre}\\
\end{align}
$\ev{\dv{(\hat{\mathcal{J}}_D^2)}{t}}$の値を経路積分を用いて計算してみる。\\
\begin{align}
  \hat{\mathcal{J}}_D^2[\bm{x}_\tau,\bm{\lambda}_\tau](t)=&\sum_{\lbrace k_1\mid t_{k_1}\le t\rbrace} D^{\nu_{k_1}}_{x_{k_1},x_{k_1-1}}\sum_{\lbrace k_2\mid t_{k_2}\le t\rbrace} D^{\nu_{k_2}}_{x_{k_2},x_{k_2-1}}\\
  =&\sum_{\lbrace k_1\mid t_{k_1}\le t\rbrace} \sum_{\lbrace k_2\mid t_{k_2}\le t\rbrace} D^{\nu_{k_1}}_{x_{k_1},x_{k_1-1}} D^{\nu_{k_2}}_{x_{k_2},x_{k_2-1}}\\
\end{align}
これを用いて、
\begin{align}
  \dv{\hat{\mathcal{J}}_D^2[\bm{x}_\tau,\bm{\lambda}_\tau](t)}{t}
  =&\sum_{k'=1}^{K}\delta(t-t_{k'})\left(\sum_{k_1=1}^{k'} \sum_{k_2=1}^{k'} D^{\nu_{k_1}}_{x_{k_1},x_{k_1-1}} D^{\nu_{k_2}}_{x_{k_2},x_{k_2-1}}-\sum_{k_1=1}^{k'-1} \sum_{k_2=1}^{k'-1} D^{\nu_{k_1}}_{x_{k_1},x_{k_1-1}} D^{\nu_{k_2}}_{x_{k_2},x_{k_2-1}}\right)\\
  =&\sum_{k'=1}^{K}\delta(t-t_{k'})\left(2D^{\nu_{k'}}_{x_{k'},x_{k'-1}}\sum_{k_1=1}^{k'-1}D^{\nu_{k_1}}_{x_{k_1},x_{k_1-1}}+(D^{\nu_{k'}}_{x_{k'},x_{k'-1}})^2\right)
\end{align}
これを用いて、経路積分を行うと
\begin{align}
  \ev{\dv{\hat{\mathcal{J}}_D^2(t)}{t}}=&\int \mathcal{D}\bm{x}_\tau \dv{\hat{\mathcal{J}}_D^2[\bm{x}_\tau,\bm{\lambda}_\tau](t)}{t}P[\bm{x}_\tau|\bm{\lambda}_\tau]\\
  =&\sum_{K=0}^{\infty}\sum_{\lbrace x_k\rbrace_{k=0}^K}\sum_{\lbrace\nu_k\rbrace_{k=1}^K}\int_{0<t_1<\cdots<t_K<\tau}\dd{t_1}\cdots \dd{t_K}\\
  \times&\sum_{k'=1}^{K}\delta(t-t_{k'})\left(2D^{\nu_{k'}}_{x_{k'},x_{k'-1}}\sum_{k_1=1}^{k'-1}D^{\nu_{k_1}}_{x_{k_1},x_{k_1-1}}+(D^{\nu_{k'}}_{x_{k'},x_{k'-1}})^2\right)\\
  \times&P(x_0,0)\exp\left(-\int_0^\tau \dd{t'}\gamma\left(x(t'),t'\right)\right)\prod_{k=1}^K R_{\nu_k}\left(x_k\mid x_{k-1};t_k\right)\\
  =&\sum_{K=0}^{\infty}\sum_{\lbrace x_k\rbrace_{k=0}^K}\sum_{\lbrace\nu_k\rbrace_{k=1}^K}\int_{0<t_1<\cdots<t_K<\tau}\dd{t_1}\cdots \dd{t_K}\sum_{k'=1}^{K}\delta(t-t_{k'})2D^{\nu_{k'}}_{x_{k'},x_{k'-1}}\sum_{k_1=1}^{k'-1}D^{\nu_{k_1}}_{x_{k_1},x_{k_1-1}}\\
  \times&P(x_0,0)\exp\left(-\int_0^\tau \dd{t'}\gamma\left(x(t'),t'\right)\right)\prod_{k=1}^K R_{\nu_k}\left(x_k\mid x_{k-1};t_k\right)\\
  +&\sum_{K=0}^{\infty}\sum_{\lbrace x_k\rbrace_{k=0}^K}\sum_{\lbrace\nu_k\rbrace_{k=1}^K}\int_{0<t_1<\cdots<t_K<\tau}\dd{t_1}\cdots \dd{t_K}\sum_{k'=1}^{K}\delta(t-t_{k'})(D^{\nu_{k'}}_{x_{k'},x_{k'-1}})^2\\
  \times&P(x_0,0)\exp\left(-\int_0^\tau \dd{t'}\gamma\left(x(t'),t'\right)\right)\prod_{k=1}^K R_{\nu_k}\left(x_k\mid x_{k-1};t_k\right)\\
  \overset{?}{=}&\sum_{\nu,x'\neq x}R_{\nu}\left(x'\mid x;t\right)P(x,t)(D^{\nu}_{x',x})^2+\order{t}
  \label{eq:JD2_dot}
\end{align}
1行目から2,3,4行目は定義を代入し、5,6,7,8行目から9行目は7,8行目は\eqref{eq:path_integral_JD}式において$D$の反対称性は用いていないため、$D_{x',x}^\nu\to(D_{x',x}^\nu)^2$として同様に計算して9行目の第1項となり、5,6行目についてはわからないが、9行目第2項になるということについてこれから示そう。\\
$\tau\to 0$について考えるので、遷移確率$R_\nu(x'|x;t)$が時間によらないとみなして計算する。
\begin{align}
  &\sum_{K=0}^{\infty}\sum_{\lbrace x_k\rbrace_{k=0}^K}\sum_{\lbrace\nu_k\rbrace_{k=1}^K}\int_{0<t_1<\cdots<t_K<\tau}\dd{t_1}\cdots \dd{t_K}\sum_{k'=1}^{K}\delta(t-t_{k'})2D^{\nu_{k'}}_{x_{k'},x_{k'-1}}\sum_{k_1=1}^{k'-1}D^{\nu_{k_1}}_{x_{k_1},x_{k_1-1}}\\
  &\times P(x_0,0)\exp\left(-\int_0^\tau \dd{t'}\gamma\left(x(t')\right)\right)\prod_{k=1}^K R_{\nu_k}\left(x_k\mid x_{k-1}\right)\\
  &=\sum_{K=0}^{\infty}\sum_{k'=1}^K\sum_{\lbrace x_k\rbrace_{k=0}^K}\sum_{\lbrace\nu_k\rbrace_{k=1}^K}\int_{0<t_1<\cdots<t_{k'-1}<t<t_{k'+1}<\cdots<t_K<\tau}\dd{t_1}\cdots\dd{t_{k'-1}}\dd{t_{k'+1}}\cdots \dd{t_K}2D^{\nu_{k'}}_{x_{k'},x_{k'-1}}\\
  &\times \sum_{k_1=1}^{k'-1}D^{\nu_{k_1}}_{x_{k_1},x_{k_1-1}}P(x_0,0)\exp\left(-\int_0^\tau \dd{t'}\gamma\left(x(t')\right)\right)\prod_{1\le k\le K,k(\neq k')} R_{\nu_{k}}\left(x_{k}\mid x_{k-1}\right)R_{\nu_{k'}}\left(x_{k'}\mid x_{k'-1}\right)\\
  &=\sum_{K=0}^{\infty}\sum_{k'=1}^K\sum_{\lbrace x_k\rbrace_{k=0}^{k'}}\sum_{\lbrace\nu_k\rbrace_{k=1}^{k'}}\int_{0<t_1<\cdots<t_{k'-1}<t}\dd{t_1}\cdots\dd{t_{k'-1}}2D^{\nu_{k'}}_{x_{k'},x_{k'-1}}\sum_{k_1=1}^{k'-1}D^{\nu_{k_1}}_{x_{k_1},x_{k_1-1}} \\
  &\times P(x_0,0)\exp\left(-\int_0^{t} \dd{t'}\gamma\left(x(t')\right)\right)\prod_{k=1}^{k'-1} R_{\nu_k}\left(x_k\mid x_{k-1}\right)R_{\nu_{k'}}\left(x_{k'}\mid x_{k'-1}\right)\\
  &\times\sum_{\lbrace x_k\rbrace_{k=k'+1}^{K}}\sum_{\lbrace\nu_k\rbrace_{k=k'+1}^{K}}\int_{t<t_{k'+1}<\cdots<t_K<\tau}\dd{t_{k'+1}}\cdots \dd{t_K}\\
  &\times\exp\left(-\int_t^\tau \dd{t'}\gamma\left(x(t')\right)\right)\prod_{k=k'+1}^K R_{\nu_k}\left(x_k\mid x_{k-1}\right)\\
  &=\sum_{K=0}^{\infty}\sum_{k'=1}^K\sum_{\lbrace x_k\rbrace_{k=0}^{k'}}\sum_{\lbrace\nu_k\rbrace_{k=1}^{k'}}\int_{0<t_1<\cdots<t_{k'-1}<t}\dd{t_1}\cdots\dd{t_{k'-1}}2D^{\nu_{k'}}_{x_{k'},x_{k'-1}}\sum_{k_1=1}^{k'-1}D^{\nu_{k_1}}_{x_{k_1},x_{k_1-1}} \\
  &\times P(x_0,0)\exp\left(-\int_0^{t} \dd{t'}\gamma\left(x(t')\right)\right)\prod_{k=1}^{k'-1} R_{\nu_k}\left(x_k\mid x_{k-1}\right)R_{\nu_{k'}}\left(x_{k'}\mid x_{k'-1}\right)\\
  &=\sum_{K=1}^{\infty}\sum_{k'=1}^K\sum_{\lbrace x_k\rbrace_{k=0}^{k'}}\sum_{\lbrace\nu_k\rbrace_{k=1}^{k'}}\int_{0<\frac{t_1}{t}<\cdots<\frac{t_{k'-1}}{t}<1}\dd{t_1}\cdots\dd{t_{k'-1}}2D^{\nu_{k'}}_{x_{k'},x_{k'-1}}\sum_{k_1=1}^{k'-1}D^{\nu_{k_1}}_{x_{k_1},x_{k_1-1}} \\
  &\times P(x_0,0)\exp\left(-\int_0^{t} \dd{t'}\gamma\left(x(t')\right)\right)\prod_{k=1}^{k'-1} R_{\nu_k}\left(x_k\mid x_{k-1}\right)R_{\nu_{k'}}\left(x_{k'}\mid x_{k'-1}\right)\\
  &=\sum_{K=1}^{\infty}\sum_{k'=1}^K t^{k'-1}\sum_{\lbrace x_k\rbrace_{k=0}^{k'}}\sum_{\lbrace\nu_k\rbrace_{k=1}^{k'}}\int_{0<t'_1<\cdots<t'_{k'-1}<1}\dd{t'_1}\cdots\dd{t'_{k'-1}}2D^{\nu_{k'}}_{x_{k'},x_{k'-1}}\sum_{k_1=1}^{k'-1}D^{\nu_{k_1}}_{x_{k_1},x_{k_1-1}} \\
  &\times P(x_0,0)\exp\left(-t\int_0^1 \dd{t''}\gamma\left(x(tt'')\right)\right)\prod_{k=1}^{k'-1} R_{\nu_k}\left(x_k\mid x_{k-1}\right)R_{\nu_{k'}}\left(x_{k'}\mid x_{k'-1}\right)\\
  &=\sum_{K=1}^{\infty}\sum_{k'=2}^K t^{k'-1}\sum_{\lbrace x_k\rbrace_{k=0}^{k'}}\sum_{\lbrace\nu_k\rbrace_{k=1}^{k'}}\int_{0<t'_1<\cdots<t'_{k'-1}<1}\dd{t'_1}\cdots\dd{t'_{k'-1}}2D^{\nu_{k'}}_{x_{k'},x_{k'-1}}\sum_{k_1=1}^{k'-1}D^{\nu_{k_1}}_{x_{k_1},x_{k_1-1}} \\
  &\times P(x_0,0)\exp\left(-t\int_0^1 \dd{t''}\gamma\left(x(tt'')\right)\right)\prod_{k=1}^{k'-1} R_{\nu_k}\left(x_k\mid x_{k-1}\right)R_{\nu_{k'}}\left(x_{k'}\mid x_{k'-1}\right)\\
  \end{align}
  \begin{align}
  &=t\sum_{K=1}^{\infty}\sum_{k'=2}^K t^{k'-2}\sum_{\lbrace x_k\rbrace_{k=0}^{k'}}\sum_{\lbrace\nu_k\rbrace_{k=1}^{k'}}\int_{0<t'_1<\cdots<t'_{k'-1}<1}\dd{t'_1}\cdots\dd{t'_{k'-1}}2D^{\nu_{k'}}_{x_{k'},x_{k'-1}}\sum_{k_1=1}^{k'-1}D^{\nu_{k_1}}_{x_{k_1},x_{k_1-1}} \\
  &\times P(x_0,0)\exp\left(-t\int_0^1 \dd{t''}\gamma\left(x(tt'')\right)\right)\prod_{k=1}^{k'-1} R_{\nu_k}\left(x_k\mid x_{k-1}\right)R_{\nu_{k'}}\left(x_{k'}\mid x_{k'-1}\right)\\
  &=\order{t}
  \label{eq:order_t}
\end{align}
上から10行目までは\eqref{eq:path_integral_JD}と同様であり、9,10行目から11,12行目は積分区間を$t$で割って変形して$K=0$の項は$\sum^0_{k'=1}$より$0$となるため$K=1$からにして、11,12行目から13,14行目は$t'_k=t_k/t$として置換積分して、13,14行目から15,16行目は$k'=1$の項は$\sum^0_{k_1=1}$より$0$となるため$k'=2$からにしている。\\
このようにして示されたので、\eqref{eq:fluctuation_relation_pre}式と\eqref{eq:JD2_dot}式を用いて、
\begin{align}
  \ev*{\Delta\hat{\mathcal{J}}_D^2}=&\tau\ev{\dv{(\hat{\mathcal{J}}_D^2)}{t}}-\tau^2\ev*{\hat{\dot{\mathcal{J}}}_D}^2+\order{\tau^3}\\
  =&\tau\left( \sum_{\nu,x'\neq x}R_{\nu}\left(x'\mid x;t\right)P(x,t)(D^{\nu}_{x',x})^2+\order{\tau} \right) -\tau^2\ev*{\hat{\dot{\mathcal{J}}}_D}^2+\order{\tau^3}\\
  =&\tau\sum_{\nu,x'\neq x}R_{\nu}\left(x'\mid x;t\right)P(x,t)(D^{\nu}_{x',x})^2+\order{\tau^2}\\
  =&\tau\Delta_D+\order{\tau^2}\\
  \label{eq:fluctuation}
\end{align}
ここで、$\Delta_D:=\sum_{\nu,x'\neq x}R_{\nu}\left(x'\mid x;t\right)P(x,t)(D^{\nu}_{x',x})^2$と定義している。\\
\eqref{eq:order_t}式と同様な計算により、$\ev*{\hat{\mathcal{J}}_D}=\order{\tau}$であることを示しておく。\\
\begin{align}
  \ev*{\hat{\mathcal{J}}_D}
  =&\sum_{K=0}^{\infty}\sum_{\lbrace x_k\rbrace_{k=0}^K}\sum_{\lbrace\nu_k\rbrace_{k=1}^K}\int_{0<t_1<\cdots<t_K<\tau}\dd{t_1}\cdots \dd{t_K}\sum_{k'=1}^{K}D^{\nu_{k'}}_{x_{k'},x_{k'-1}}\\
  &\times P(x_0,0)\exp\left(-\int_0^\tau \dd{t'}\gamma\left(x(t')\right)\right)\prod_{k=1}^K R_{\nu_k}\left(x_k\mid x_{k-1}\right)\\
  =&\sum_{K=1}^{\infty}\sum_{\lbrace x_k\rbrace_{k=0}^K}\sum_{\lbrace\nu_k\rbrace_{k=1}^K}\int_{0<\frac{t_1}{\tau}<\cdots<\frac{t_K}{\tau}<1}\dd{t_1}\cdots \dd{t_K}\sum_{k'=1}^{K}D^{\nu_{k'}}_{x_{k'},x_{k'-1}}\\
  &\times P(x_0,0)\exp\left(-\int_0^\tau \dd{t'}\gamma\left(x(t')\right)\right)\prod_{k=1}^K R_{\nu_k}\left(x_k\mid x_{k-1}\right)\\
  =&\sum_{K=1}^{\infty}\tau^K\sum_{\lbrace x_k\rbrace_{k=0}^K}\sum_{\lbrace\nu_k\rbrace_{k=1}^K}\int_{0<t'_1<\cdots<t'_K<1}\dd{t'_1}\cdots \dd{t'_K}\sum_{k'=1}^{K}D^{\nu_{k'}}_{x_{k'},x_{k'-1}}\\
  &\times P(x_0,0)\exp\left(-\tau\int_0^1 \dd{t'}\gamma\left(x(\tau t')\right)\right)\prod_{k=1}^K R_{\nu_k}\left(x_k\mid x_{k-1}\right)\\
  =&\order{\tau}
\end{align}
上から1,2行目から3,4行目は$K=0$の項が$\sum_{k'=1}^0=0$より$K=1$からにして、積分範囲を$\tau$で割って書き換えている。3,4行目から5,6行目は$t'_k=t_k/\tau$として置換積分している。\\
これらの計算とエントロピー生成に関して$\sigma=\tau\dot{\sigma}+\order{\tau^2}$(示すべきか)が成り立つことと\eqref{eq:JD2_dot}式と\eqref{eq:fluctuation}式を用いて、定常状態の熱力学不定性関係\eqref{eq:thermal_uncertain_steady}に代入すると
\begin{align}
  \tau \dot{\sigma}\ge&2\frac{\tau^2\ev*{\hat{\dot{\mathcal{J}}}_D}^2}{\tau \Delta_D}\\
  \dot{\sigma}\ge&2\frac{\ev*{\hat{\dot{\mathcal{J}}}_D}^2}{\Delta_D}\\
\end{align}
このように求められる。

\end{document}